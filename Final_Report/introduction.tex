\chapter{Introduction}

\section{Introduction \& Motivation }

Main Problem \\
Optimise sensor positions. Common Problem. \\



Solution to the Problem
Investigate and Explore the solution proposed by \citet{krause_near-optimal_2008} on near-optimal optimisation \\
\\ > Details on the Process
Gaussian Processes \\


Applications : \\
Application to the Problem of Data Assimilation. Assimilate data on a limited number of points. Use the optimal sensors for the Data Assimilation and see if the results are better than random.  Can be used on other sets.  \\

Challenge is the size of of the candidate set. \\


 



\todo{Write Intoduction}


\todo{Write Problem Statement : Why we need that ? What solution ?
}


\section{Aim of the project }


\paragraph{Structure of the Project}

\todo{Explain the steps of the project report}


\paragraph{Project Proposal:}
Gaussian processes (GP) have been widely used since the 1970s in the fields of geostatistics and meteorology. Current applications are in diverse fields including sensor placement
In this project, we propose the employment of a GP model to calculate the optimal spatial positioning of sensors to study and collect air pollution data in big cities. We will then validate the results by means of a data assimilation software with the data at the proposed positions.

\paragraph{Dataset:} London South Bank University (LSBU) air pollution data (velocity, tracer)


