\documentclass[11pt,a4paper]{article}
\usepackage[american]{babel}
\usepackage[utf8x]{inputenc}
\usepackage{url}
\usepackage{amsmath}
\usepackage{amsfonts}
\usepackage{graphicx}
\graphicspath{{images/}}
\usepackage{parskip}
\usepackage{fancyhdr}
\usepackage{wrapfig}
\usepackage{indentfirst}
\usepackage{vmargin}
\usepackage[font=small,labelfont=bf]{caption}
\usepackage[colorinlistoftodos]{todonotes}
\usepackage{indentfirst}
\usepackage{hyperref}
\usepackage{fullpage}
\usepackage{booktabs}
\usepackage{caption}
\usepackage{enumerate}
\usepackage{indentfirst}
\usepackage{color}
\usepackage{verbatim}
\usepackage{capt-of}
\usepackage[]{algorithm2e}
\usepackage{float}
\usepackage{multirow}
\usepackage{framed}
\usepackage{listings}
\usepackage{color}
\usepackage{datetime}
\usepackage{url}
\usepackage[numbers]{natbib}
\usepackage{epstopdf}
\usepackage{subfig}

\numberwithin{equation}{section}

\renewcommand{\vec}[1]{\mathbf{#1}}

 
\definecolor{codegreen}{rgb}{0,0.6,0}
\definecolor{codegray}{rgb}{0.5,0.5,0.5}
\definecolor{codepurple}{rgb}{0.58,0,0.82}
\definecolor{backcolour}{rgb}{0.95,0.95,0.92}
 
\lstdefinestyle{mystyle}{
    backgroundcolor=\color{backcolour}, 
    commentstyle=\color{codegreen},
    keywordstyle=\color{magenta},
    numberstyle=\tiny\color{codegray},
    stringstyle=\color{codepurple},
    basicstyle=\footnotesize,
    breakatwhitespace=false,         
    breaklines=true,                 
    captionpos=b,                    
    keepspaces=true,                 
    numbers=left,                    
    numbersep=5pt,                  
    showspaces=false,                
    showstringspaces=false,
    showtabs=false,                  
    tabsize=2
}
 
\lstset{style=mystyle}
\setmarginsrb{1.5 cm}{1.5 cm}{1.5 cm}{1.5 cm}{0.5 cm}{1 cm}{0.5 cm}{1 cm}

\title{Gaussian Processes for Optimal Sensor Position \\
Master Project \\
Background and Progress Report 
}							% Title
\author{Adrian Löwenstein}								% Author
\date{\today}											% Date

\makeatletter
\let\thetitle\@title
\let\theauthor\@author
\let\thedate\@date
\makeatother


\pagestyle{fancy}
\fancyhf{}
\rhead{\theauthor}
\lhead{Gaussian Processes for Optimal Sensor Position}
\cfoot{\thepage}

\newdate{date}{22}{06}{2018}

\begin{document}
\maketitle
\pagebreak

\section{Project Description}

\subsection{Summary}
Gaussian processes (GP) have been widely used since the 1970’s in the fields of geostatistics and meteorology. Current applications are in diverse fields including sensor placement
In this project, we propose the employment of a GP model to calculate the optimal spatial positioning of sensors to study and collect air pollution data in big cities. We will then validate the results by means of a data assimilation software with the data at the proposed positions.

\subsection{Data}
London South Bank University (LSBU) air pollution data (velocity, tracer)

\section{Literature Review}

\subsection{The MAGIC Project}

\subsection{Modelling Spacial Data with Gaussian Process}

\subsubsection{ ... }

\subsubsection{Estimating the Covariance Function}

\subsection{Optimising the Sensors Locations}

\subsection{Data Assimilation}

\subsubsection{Definitions}

\subsubsection{DA in the MAGIC Project}

\section{Progress Report}

\subsection{Learning the GP}
\subsubsection{Data Points }
In our case we have data available from the simulation and from the 

\subsubsection{Covariance Matrix}


\subsection{Optimisation of Sensors Locations}

\subsection{Validat}

\begin{itemize}
	\item Estimate the performance at the current Locations
	\item Propose new placements 
\end{itemize}


\end{document}